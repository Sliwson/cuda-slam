\documentclass[titlepage]{article}
\usepackage[T1]{fontenc}
\usepackage[utf8]{inputenc}
\usepackage{biblatex}
\usepackage{titling, lipsum}
\usepackage{graphicx}
\usepackage{geometry}

\usepackage{hyperref} 

\geometry{
 a4paper,
 total={170mm,257mm},
 margin=1in
}

\addbibresource{bibliography.bib}

\begin{document}
\begin{titlepage}
	{\centering
	{\scshape\huge Comparison of SLAM methods with CUDA implementation \par}
	\vspace{1cm}
	{\scshape\Large Course: Research project - GPU algorithms \par}}
	
	\vspace{1cm}
	\noindent\textbf{Coordinator}: Krzysztof Kaczmarski\\
	\textbf{Authors}: Michał Rogala, Szymon Stasiak, Mateusz Śliwakowski\\
	\textbf{Description}: Short description\\
	\textbf{Code repository}: \href{https://github.com/Sliwson/cuda-slam}{https://github.com/Sliwson/cuda-slam}\\
	\textbf{Code license}: MIT\\
	\textbf{Input files}: Point clouds as .obj files\\

	\vfill
	{\large \today \par}
\end{titlepage}

\tableofcontents
\newpage

\section{Report goals}
What we will know after reading this document?
\section{Problem statement}
Description of the problem, motivating example, killer application, etc.
\section{Computational method}
The most important ideas in the algorithm necessary to understand how it works.
Are there any differences from other methods?
Are there any novel ideas?
\section{Program architecture}
Short description on the modules of the system, requirements, dependencies, etc.
\section{Input data description}
File formats, api used to read, dependencies, data sources, etc.
Is there any institution providing the input data?
References to public databases.
\section{Execution, configuration and user guide}
How can one replicate the experiments?
Any relevant information and runtime howtos.
\section{Description of the results}
Performance of the system for different input data/parameters.
How we understand the results and why are they correct?
Can we formulate any conclusions from the experiments?
\section{Remarks}
Any remarks to the results and methods.
\section{Future works}
What do we want to do in future, improvements.

This is bibliography sample - it was noted in\cite{ms-paper1}.

\newpage
\printbibliography

\end{document}

